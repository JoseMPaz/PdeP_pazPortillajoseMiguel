\begin{titlepage}

\thispagestyle{empty}

\begin{center}
\includegraphics[scale=0.5]{./figuras/logo_utn}\\
\hfill \newline
\large{\textsc{Resumen Paradigma Funcional}}\\
\large{\textsc{Facultad Regional Buenos Aires}}\\
\large{\textsc{Universidad Tecnologica Nacional}}\\
\end{center}

% \newline
\begin{center}
\LARGE{\textsc{Paradigmas de Programación -- $K2032$}}\\
\hfill \newline
\huge{Trabajos Prácticos}
\end{center}

\vspace{2cm}



\begin{center}
	\begin{tabular}{lc}
		PAZ PORTILLA, José Miguel & \ \ \ 2028244 \\
		\texttt{\href{mailto:jpazportilla@frba.utn.edu.ar}{jpazportilla@frba.utn.edu.ar}}\\
	\end{tabular}
\end{center}

\vspace{1cm}
\begin{center}
\large{\today}
\end{center}

\end{titlepage}

%
% Pongo el índice en una página aparte:
%
{
  \hypersetup{linkcolor=black}
  \tableofcontents
}
% \tableofcontents
\thispagestyle{empty}
\newpage
%
% Hago que las páginas se comiencen a contar a partir de aquí:
%
\setcounter{page}{1}
